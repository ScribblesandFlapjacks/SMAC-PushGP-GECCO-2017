\section{Introduction}
\label{sec:introduction}

Set up SMAC, what we did with it, and what we learned.

\todo{The following could be moved to Background if we think it makes more
	sense there.}
It is important to keep in mind that there is little in this paper that is
specific to SMAC or its details or, for that matter, Clojush or its details. 
There are many approaches to parameter
tuning, and presumably similar results could have been acquired with many of
them. What matters the most here is that:
\begin{itemize}
	\item SMAC was able to discover parameter settings that 
	\emph{substantially} improved the performance
	of Clojush on the Replace-space-with-newline problem.
	\item Applying those optimized parameter settings to a variety of other
	problems let to very mixed results, suggesting that those parameter
	settings were very specific to the particular problem used for the SMAC
	search.
\end{itemize}
The fact that SMAC was able to discover settings that improved
performance on the Replace-space-with-newline problem by such a wide margin
is quite impressive. The fact that those parameter settings don't appear to
generalize is in no way the ``fault'' of SMAC, and are likely much more 
about how we used the tool than any property of the tool itself.

\section{Background}
\label{sec:background}

\begin{itemize}
	\item The citation the SMAC people ask us to use is~\cite{HutHooLey11-SMAC}.
	\item We should also probably cite their GECCO 2013 
	paper~\cite{hutter2013evaluation} as a prior example of using this in EC.
\end{itemize}

This is pretty fuzzy to me (Nic). We don't need to say a lot about \emph{how}
SMAC works, but we do need to talk about \emph{what it does}. Maybe that just
gets blended into the introduction?

Somewhere we'll presumably need to acknowledge that (a) there's been lots of
work historically on parameter optimization (e.g., early work on things like
population sizes and mutation rates in GAs) and (b) SMAC is by no means the
only tool of its kind out there. It would be great if someone could do some
lit review on that latter question in particular. (A few early examples would
be sufficient to make the first point; the latter is more politically charged.)

We also need to provide at least a minimal overview of Push/Plush/Clojush so
we can explain the meaning/action of the the parameters we chose to tune. 

\section{Tuning parameters for Replace-space-with-newline}
\label{sec:tuningRSWN}

% Describe the August work in Morris.

\todo{The following is almost certainly overly chatty and historical (vs.
	scientific), and will probably need to be cleaned up. It's just how
	I find myself thinking about it, and it seemed better to write *something*
	than sweat those bullets right now. -- Nic}

After hearing about SMAC at Holger Hoos's keynote at GECCO in July, 2016, we
decided to see if we could use SMAC to optimize parameters for Clojush, a
Clojure implementation of PushGP. PushGP is a stack-based genetic programming
system that has been demonstrably successful on a wide range of software
synthesis problems~\cite{Helmuth:2015:GECCO}. Like most ``industrial
strength'' EAs, Clojush has dozens of parameters that need to be set by the
practicioner, ranging from ``standard'' parameters like the population size
and the maximum number of generations to more Push-specific parameters\todo{
	Are these \emph{Push}-specific, or \emph{Plush}-specific? Do we want to
	get into those weeds? Do we want/need to bring Plush into it at all?}
such as \emph{alignment deviation} and \emph{uniform close mutation} (described
in Section~\ref{sec:background} above).\todo{Which means we need to have actually described them up above.}

After some initial work to set SMAC up to work with Clojush, and convince
ourselves that SMAC was in fact likely to perform some useful parameter
optimization, we set up three extended SMAC runs, each working independently
to optimize parameters for the Replace-space-with-newline software synthesis 
problem.\todo{Do we need to describe RSWN? The other 4? Where? In the Background section?}
This optimization depended on two important sets of choices:
\begin{itemize}
	\item The parameters to be optimized, and over what ranges
	\item The settings applied to SMAC, such as how long to set it search
\end{itemize}
These will be described below, along with the results of that optimization.

\subsection{Parameters optimized}
\label{sec:parametersOptimized}

Clojush exposes dozens of parameters that could potentially be optimized, but
adding parameters effectively increases the dimensionality and size of the
search space SMAC is required to explore. Thus exposing every possible parameter
seemed unwise, especially in a first experiment with the tool. We chose then
to focus primarily on parameters that controlled the genetic operators, along
with the population size and the selection mechanism. The parameters we chose 
to explore, the ranges SMAC should explore for each parameter, and the 
default initial values, are all listed in Table~\ref{tab:clojushParameters}.
In each case the default initial value was the same value used for all the
experiments in~\cite{Helmuth:2015:GECCO}.

\begin{table}
	\begin{center}
	\begin{tabular}{l@{\quad} @{\enskip}l@{\quad} @{\enskip}r@{\quad} @{\enskip}l}
		Parameter & Range & Default & Modifiers \\
		\hline
		Population size & $[1, 30K]$ & 1,000 & integer, log \\
		% Maximum generations & $[1, 30K]$ & $30,000/\textrm{population size}$ \\
		Selection method & tournament or & lexicase & categorical \\
		& \quad lexicase \\
		\hline
		Alternation prob. & $[0, 1]$ & 0.2 \\
		Uniform mutation prob. & $[0, 1]$ & 0.2 \\
		Uniform close mutation prob. & $[0, 1]$ & 0.1 \\
		(Alternation, Uniform mutation) prob. & $[0, 1]$ & 0.5 \\
		\hline
		Alternation rate & $[0, 1]$ & 0.01 \\
		Alignment deviation & $[0, 400]$ & 10 & integer \\
		\hline
		Uniform mutation rate & $[0, 1]$ & 0.1
	\end{tabular}
	\end{center}
	\caption{Clojush parameters optimized by SMAC for the Replace-space-with-newline problem, along with their ranges, 
		default values, and modifiers. See Section~\ref{sec:parametersOptimized} for additional details.}
	\label{tab:clojushParameters}
\end{table}

\todo{Should ``Maximum generations'' be in the table? It was computed
		and never actually exposed to SMAC, so it's an odd duck.}
The first two parameters in Table~\ref{tab:clojushParameters} (population size
and selection method) are fairly ``generic'' parameters. Population size, 
however, presented an interesting issue, because while we wanted SMAC to be
explore the impact of population size, we needed to maintain a consistent
computational budget; otherwise SMAC would presumably discover the ``obvious''
fact that it is almost always desirable to provide more computational resources to the underlying Clojush search. We wanted to make the runs comparable to those
reported in~\cite{Helmuth:2015:GECCO}, all of which used a population size of
1,000 for 300 generations, meaning that at most 30K individuals were evaluated 
in the course of a run.\footnote{Clojush runs there, and in this work, 
	were terminated early when a solution was found, so some runs used
	considerably less than their full budget of 30K evaluations.} To provide
a similar constraint here, we allowed SMAC to explore the full range of 
$[1, 30K]$ for the population size, and computed the maximum number of
generations as
\[
	\left \lfloor{\frac{30,000}{\textrm{population size}}}\right \rfloor
\]
This ensures that no Clojush run explored by SMAC would be allowed 
\emph{more} than 30K individual evaluations, although some choices of 
population size would be ``penalized'' by being allocated substantially fewer. 
Any population size over 15K, for example, would get only one generation, which 
means a population size just over 15K would get just over half the evaluations 
that a ``full'' run would. Our hope was that SMAC would be able to incorporate
these impacts in its parameter search without any great difficulty.

We also used the SMAC option of declaring the population size to be an
integer value (so SMAC wouldn't explore non-integral values for that parameter),
and we applied SMAC's \emph{log} modifier to the population size parameter.
The SMAC \emph{log} modifier is used to indicate parameters that naturally
vary on a logrithmic scale. This means that from SMAC's perspective a change
in population size from 10 to 100 has the same magnitude as a change from 1,000 to
10,000. We made the choice on the assumption that while there might be value
in exploring small changes to small population sizes (e.g., from 10 to 12),
there was unlikely to be value in exploring small changes for large sizes
(e.g., from 20,000 to 20,002).

The selection mechanism parameter used SMAC's \emph{categorical} qualifier to
indicate that it's set of values came from a set of discrete, non-numerical
options. In this case we allowed SMAC to explore two choices for the selection
mechanism: traditional tournament selection (with a tournament size of 7) and
lexicase selection~\cite{lexicase:things}. Earlier 
work~\cite{Helmuth:2015:GECCO} had shown that lexicase selection was
significantly more effective than tournament selection, at least with a set of 
default parameter values that hadn't been particularly tuned for either
selection mechanism. By letting SMAC explore both selection mechanisms while
tuning several other parameters, it was possible that SMAC would discover
that tournament selection might outperform lexicase selection when combined
with a particular collection of parameter settings.

\todo{Do we want/need this big footnote about normalization and skew?}
The next four parameters in Table~\ref{tab:clojushParameters} indicate the
relative likelihood of each of the four genetic operators. These need to sum
to 1 (since every individual must be created via \emph{some} genetic operator),
so these were normalized before being passed from SMAC to Clojush by dividing
each by the sum of the four.\footnote{This arguably inflates the dimensionality of the
search space for SMAC; from SMAC's perspective these parameters provide four
degrees of freedom, when in reality there are only three since values for any
three determine the value of the fourth. The obvious ways of addressing this
seemed that they might badly skew the search space, however. If, for example,
we computed the last based on the sum of the first three, most choices for the
first three would be illegal (they alone would sum to more than 1), and
proportionally very few would allow for a relatively large value for the
remaining parameter. So we chose to present SMAC with four apparently 
independent parameters and normalize.}

The next two parameters in Table~\ref{tab:clojushParameters} (alternation
rate and alignment deviation) are control the behavior of the alternation
operator that plays the role of crossover in these runs. It's important to
note that very large values for alignment deviation make it increasingly
likely that an alternation event will move \todo{Say something about moving off the front or the back off the other parent, blah, blah.}

\subsection{SMAC settings}
\label{sec:SMACsettings}

\subsection{SMAC results}
\label{sec:SMACresultsRSWN}

\section{Applying those parameters to other problems}
\label{sec:applyingToOtherProblems}

These are the parameters I'm using, which are approximately those taken from Nic's SMAC optimizations on RSWN:

\begin{verbatim}
:uniform-mutation-rate 0.18
:alignment-deviation 120
:genetic-operator-probabilities {:alternation 0.04
                                 :uniform-mutation 0.36
                                 :uniform-close-mutation 0.06
                                 [:alternation :uniform-mutation] 0.54}
\end{verbatim}

All other pushgp parameters are set to default, including population size of 1000.

\begin{table}[t]
\centering
\caption{Results with SMAC tuned parameters}
\label{table:results}
%\rowcolors{3}{Gray}{white}
\begin{tabular}{l r r}
\toprule
\textbf{Problem} & \textbf{Standard Params} & \textbf{SMAC Params} \tabularnewline
\midrule
Double Letters	& 0 & 6 \tabularnewline
Replace Space with Newline & 54 & 91 \tabularnewline
String Lengths Backwards & 68 & 75 \tabularnewline
X-Word Lines & 17 & 3 \tabularnewline
Syllables & 22 & 17 \tabularnewline
\bottomrule
\end{tabular}
\end{table}

Here's text that I wrote on Sakai:

So far, only the RSWN runs are done. For those, 95/100 runs found successful programs, of which 91 generalized to the unseen test data. I believe @mcphee's setup of SMAC is using success on the training data to guide it. So, even though 95\% is slightly worse than his 98\% result, it is almost as good, and is significantly better than the best result I've ever seen on the problem (59\%, with our previously standard parameters).

Now for the bad news. 80 runs have finished on the Syllables problem, and only 16/80 have found solutions (all of which generalized). If this rate continues, it will be almost equivalent to 22 solutions with our standard settings.

So, I think we're seeing that the improvements SMAC made to parameters when using RSWN don't necessarily translate to more success on other problems. It's still worthwhile to see if those parameters help on a handful of other problems, in case Syllables is simply weird in a different way.

So, the runs for String Lengths Backwards, Syllables, and Double Letters are done with the RSWN SMAC settings. The takeaway: all results are almost identical to runs with our standard settings, and none are significantly different!

So, the settings above area really good at solving RSWN, but don't help at all with a handful of other problems! I don't know what to make of this! :confused: I'm not sure why those settings would be specifically good for RSWN, and no better or worse for other problems. I wonder if there's something about RSWN that makes it susceptible to high mutation rates and high alignment deviation.

At this point, I think it would certainly be interesting to run SMAC on either other single problems or on multiple problems at once (or maybe both).

\section{Discussion}
\label{sec:discussion}

Talk about what we learned and what this means

A piece from @thelmuth at Discourse that we might want to use:
\begin{quote}
	Here's a simpler version of my first explanation: alternation makes viable children more often, so you need less of it per event that you want to be successful. Uniform mutation makes viable children less often, so you need more of it per event that you want to be successful. Uniform close mutation makes viable children the most often, so you need least of it per event you want to be successful. Then, in the ancestors of the solution, you get the percents of the events you want to be successful, not the percent of the events you have in total.
\end{quote}

\section{Future work}
\label{sec:futureWork}

\begin{itemize}
	\item Use SMAC to tune parameters on the other four problems one at a time, and see how that works on the other problems.
	\item Use SMAC to tune on (sub)sets of the problems, like all 5, or on subsets that seem similar (or different) based on the individual tuning.
	\item etc.
\end{itemize}

\section{Conclusions}
\label{sec:conclusion}

Evolution is complicated!

