% This is LLNCS.DEM the demonstration file of
% the LaTeX macro package from Springer-Verlag
% for Lecture Notes in Computer Science,
% version 2.4 for LaTeX2e as of 16. April 2010
%
\documentclass{llncs}
%
\usepackage{makeidx}  % allows for indexgeneration
\usepackage{todonotes}
%
\begin{document}

\mainmatter              % start of the contributions
%
\title{SMAC, PushGP, and parameter (over) optimization}
%
%\titlerunning{SMAC, PushGP, and parameter (over) optimization}  % abbreviated title (for running head)
%                                     also used for the TOC unless
%                                     \toctitle is used
%

\todo[inline]{No idea what order we do the authors here. May end up depending on how the paper eventually gets written.}

\todo[inline]{Need to make sure we update the authorrunning and the tocauthor and the institute entries when we figure out who the authors actually are, etc.}

\todo[inline]{I think we'll eventually have to do something about the indexing?}

\author{Nic\inst{1} \and Tom\inst{2} \and Lee\inst{3} \and Whoever Else In Whatever Order Makes Sense}
%\author{Ivar Ekeland\inst{1} \and Roger Temam\inst{2}
%Jeffrey Dean \and David Grove \and Craig Chambers \and Kim~B.~Bruce \and
%Elsa Bertino}
%
\authorrunning{Ivar Ekeland et al.} % abbreviated author list (for running head)
%
%%%% list of authors for the TOC (use if author list has to be modified)
\tocauthor{Ivar Ekeland, Roger Temam, Jeffrey Dean, David Grove,
Craig Chambers, Kim B. Bruce, and Elisa Bertino}
%
\institute{University of Minnesota -- Morris, Morris MN 56267, USA, \\
	\email{mcphee@morris.umn.edu}
	\and
	Washington and Lee University, Lexington, VA 24450, USA, \\
	\email{helmutht@wlu.edu}
	\and
	Hampshire College, Amherst, MA 01002, USA, \\
	\email{lspector@hampshire.edu}
}
%\institute{Princeton University, Princeton NJ 08544, USA,\\
%\email{I.Ekeland@princeton.edu},\\ WWW home page:
%\texttt{http://users/\homedir iekeland/web/welcome.html}
%\and
%Universit\'{e} de Paris-Sud,
%Laboratoire d'Analyse Num\'{e}rique, B\^{a}timent 425,\\
%F-91405 Orsay Cedex, France}

\maketitle              % typeset the title of the contribution

\begin{abstract}
The abstract should summarize the contents of the paper
using at least 70 and at most 150 words. It will be set in 9-point
font size and be inset 1.0 cm from the right and left margins.
There will be two blank lines before and after the Abstract. \dots
\keywords{computational geometry, graph theory, Hamilton cycles}
\todo[inline]{need to write an abstract}
\todo[inline]{Need to sort out keywords}
\end{abstract}

\section{Indtroduction}

Set up SMAC, what we did with it, and what we learned.

\todo{The following could be moved to Background if we think it makes more
	sense there.}
It is important to keep in mind that there is little in this paper that is
specific to SMAC or its details or, for that matter, Clojush or its details. 
There are many approaches to parameter
tuning, and presumably similar results could have been acquired with many of
them. What matters the most here is that:
\begin{itemize}
	\item SMAC was able to discover parameter settings that 
	\emph{substantially} improved the performance
	of Clojush on the Replace-space-with-newline problem.
	\item Applying those optimized parameter settings to a variety of other
	problems let to very mixed results, suggesting that those parameter
	settings were very specific to the particular problem used for the SMAC
	search.
\end{itemize}
The fact that SMAC was able to discover settings that improved
performance on the Replace-space-with-newline problem by such a wide margin
is quite impressive. The fact that those parameter settings don't appear to
generalize is in no way the ``fault'' of SMAC, and are likely much more 
about how we used the tool than any property of the tool itself.

\section{Background}

This is pretty fuzzy to me (Nic). We don't need to say a lot about \emph{how}
SMAC works, but we do need to talk about \emph{what it does}. Maybe that just
gets blended into the introduction?

Somewhere we'll presumably need to acknowledge that (a) there's been lots of
work historically on parameter optimization (e.g., early work on things like
population sizes and mutation rates in GAs) and (b) SMAC is by no means the
only tool of its kind out there. It would be great if someone could do some
lit review on that latter question in particular. (A few early examples would
be sufficient to make the first point; the latter is more politically charged.)

\section{Tuning parameters for Replace-space-with-newline}

Describe the August work in Morris.

\section{Applying those parameters to other problems}

Describe Tom's subsequent work applying those parameters to 4 other
problems.

\section{Discussion}

Talk about what we learned and what this means

\section{Future work}

\begin{itemize}
	\item Use SMAC to tune parameters on the other four problems one at a time, and see how that works on the other problems.
	\item Use SMAC to tune on (sub)sets of the problems, like all 5, or on subsets that seem similar (or different) based on the individual tuning.
	\item etc.
\end{itemize}

\section{Conclusions}

Evolution is complicated!

%
% ---- Bibliography ----
%
\begin{thebibliography}{5}
%
\bibitem {clar:eke}
Clarke, F., Ekeland, I.:
Nonlinear oscillations and
boundary-value problems for Hamiltonian systems.
Arch. Rat. Mech. Anal. 78, 315--333 (1982)

\bibitem {clar:eke:2}
Clarke, F., Ekeland, I.:
Solutions p\'{e}riodiques, du
p\'{e}riode donn\'{e}e, des \'{e}quations hamiltoniennes.
Note CRAS Paris 287, 1013--1015 (1978)

\bibitem {mich:tar}
Michalek, R., Tarantello, G.:
Subharmonic solutions with prescribed minimal
period for nonautonomous Hamiltonian systems.
J. Diff. Eq. 72, 28--55 (1988)

\bibitem {tar}
Tarantello, G.:
Subharmonic solutions for Hamiltonian
systems via a $\bbbz_{p}$ pseudoindex theory.
Annali di Matematica Pura (to appear)

\bibitem {rab}
Rabinowitz, P.:
On subharmonic solutions of a Hamiltonian system.
Comm. Pure Appl. Math. 33, 609--633 (1980)

\end{thebibliography}

\clearpage
\addtocmark[2]{Author Index} % additional numbered TOC entry
\renewcommand{\indexname}{Author Index}
\printindex
\clearpage
\addtocmark[2]{Subject Index} % additional numbered TOC entry
\markboth{Subject Index}{Subject Index}
\renewcommand{\indexname}{Subject Index}
\input{subjidx.ind}
\end{document}
